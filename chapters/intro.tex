\chapter{Introduction}
\label{chap:Introduction}

\begin{flushleft}
Bone fractures are a significant health concern globally with 178 million new cases and 455 million prevalent cases of acute or long term 
symptoms in 2019 alone. [1] Elbow fractures in particular are one of the most common types of bone fractures among children, accounting for over 
10\% of paediatric fractures and up to 5\% of fractures among adults. Accurate and timely diagnosis of elbow fractures is important to avoid 
potential complications such as joint stiffness, malunion and vascular injury. [2] However, intepretation of elbow X-rays is a difficult task.
Some parts of the anatomy of the elbow are obscured depending on the perspective of the X-ray (anteroposterior, lateral or oblique) and 
non displaced elbow fractures are difficult to spot on X-rays [3]. Diagnosis of paediatric elbow fractures is especially difficult due to the
differences in anatomy and injury patterns from the adult elbow such as the presence of ossification centres [4].

Recent advances in machine learning and deep learning have allowed for the use of automated systems to quickly detect fractures from X-rays. 
These often rely on convolutional neural networks (CNN) which are able to automatically learn complex features and patterns from images
and have been shown to have excellent performance in fracture detection, on par with or exceeding trained radiologists [5]. The use of these
automated systems could help support doctors in accurately diagnosing elbow fractures, especially in emergency settings and for paediatric elbow 
fractures which may be initially reviewed by radiologists who do not specialise in paediatrics [6,7]. 

One challenge faced by CNN models as well as other deep learning models is the amount of available data which is often limited in medical domains. 
To address this, deep learning models can be pretrained on other datasets which are related to the target dataset. In the case of diagnosing elbow 
fractures, this can include pretraining on X-rays of other bones such as the shoulder, or even X-rays of other body parts such as chest X-rays [8].
Other techniques to improve the performance of deep learning models include the use of attention mechanisms which allow the model to focus on 
important parts of an image and ensembling techniques to combine the output of multiple models.

Other than CNN models, another area that shows potential is the use of vision language foundation models. Foundation models are pretrained
on huge multimodal datasets which contain both image and text and cover many different domains. This allows the models to perform a wide
variety of tasks with or without task specific finetuning. Foundation models developed for general purposes may not perform well in the medical 
domain due to the lack of medical data in their training sets. However, there have also been foundation models which been specifically trained 
for medical domains to perform tasks such as generating medical reports or segmenting images [9]. Although foundation models are much larger
and more computationally expensive than CNNs, the same model could potentially be used for many different medical tasks including those which
combine image and text data which is a significant advantage over a CNN which is more narrow.

This project aims to compare and evaluate the performance of CNNs and vision language foundation models on detecting elbow fractures.
\end{flushleft}

% 1. https://pubmed.ncbi.nlm.nih.gov/34723233/

% 2. https://www.ncbi.nlm.nih.gov/books/NBK441976/ 

% 3. https://www.sciencedirect.com/science/article/abs/pii/S0735675706000258

% 4. https://ajronline.org/doi/full/10.2214/AJR.10.7314

% 5. https://pmc.ncbi.nlm.nih.gov/articles/PMC10826962/#sec002

% 6. https://pubmed.ncbi.nlm.nih.gov/30348771/

% 7. https://www.cambridge.org/core/journals/canadian-journal-of-emergency-medicine/article/determining-the-clinical-significance-of-errors-in-pediatric-radiograph-interpretation-between-emergency-physicians-and-radiologists/E25C5BDFC9E08F894CEA7755A01591E4

% 8. https://www.mdpi.com/2072-6694/15/15/4007

% 9. https://arxiv.org/pdf/2406.10729

