\begin{abstract}
\noindent % Prevents indenting at the start of the paragraph
Elbow fractures are a common type of bone fracture among children, and can lead to serious complications if 
not quickly diagnosed and treated. Deep learning models such as convolutional neural networks have proven effective 
at detecting elbow fractures, matching the performance of trained radiologists. However, the performance of these models is 
hampered by a lack of training data. In this project, we developed a framework to train attention based CNN models to detect elbow fractures. 
The framework was tested with 3 CNNs: ConvNeXt, EfficientNetV2 and MobileNetV3. In addition, we finetuned 2 medical vision language models, MedGemma and 
LLaVA-Rad, and evaluated their performance against the CNNs. The CNN models used were modified with the addition of convolutional block attention modules and 
pretrained on the MURA dataset before training on the target dataset. Subsequently, feature fusion was used to train an ensemble model. The vision language models 
were finetuned on the same training dataset using Low Rank Adaptation. The attention based CNN models showed a significant improvement over the original models,
ranging from 1.5\% to 3.1\% improvement in accuracy while the ensemble model achieved the highest recall of 94.3\% and second highest precision of 91.9\%. Medgemma had 
poor performance with an accuracy of only 88.1\%, below that of the attention based CNN models. However LLaVA-Rad had an accuracy of 91.5\% which surpassed 
2 of the CNN models and had a high recall of 93.8\%, demonstrating high potential for further work.
\end{abstract}
\pagebreak
